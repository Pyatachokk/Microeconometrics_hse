\documentclass[a4paper,12pt]{article}

%%% Работа с русским языком
\usepackage{cmap}					% поиск в PDF
\usepackage{mathtext} 				% русские буквы в формулах
\usepackage[T2A]{fontenc}			% кодировка
\usepackage[utf8]{inputenc}			% кодировка исходного текста
\usepackage[english,russian]{babel}	% локализация и переносы

%%% Дополнительная работа с математикой
\usepackage{amsfonts,amssymb,amsthm,mathtools} % AMS
\usepackage{amsmath}
\usepackage{icomma} % "Умная" запятая: $0,2$ --- число, $0, 2$ --- перечисление

\usepackage[left = 2cm, right = 2cm, top = 2cm, bottom = 2cm]{geometry}

\usepackage{booktabs}
\usepackage{dcolumn}

%% Номера формул
%\mathtoolsset{showonlyrefs=true} % Показывать номера только у тех формул, на которые есть \eqref{} в тексте.

%% Шрифты
\usepackage{euscript}	 % Шрифт Евклид
\usepackage{mathrsfs} % Красивый матшрифт

%% Свои команды
\DeclareMathOperator{\sgn}{\mathop{sgn}}

%% Перенос знаков в формулах (по Львовскому)
\newcommand*{\hm}[1]{#1\nobreak\discretionary{}
	{\hbox{$\mathsurround=0pt #1$}}{}}

%%% Работа с картинками
\usepackage{graphicx}  % Для вставки рисунков
\graphicspath{{images/}{images2/}}  % папки с картинками
\setlength\fboxsep{3pt} % Отступ рамки \fbox{} от рисунка
\setlength\fboxrule{1pt} % Толщина линий рамки \fbox{}
\usepackage{wrapfig} % Обтекание рисунков и таблиц текстом

%%% Работа с таблицами
\usepackage{array,tabularx,tabulary,booktabs} % Дополнительная работа с таблицами
\usepackage{longtable}  % Длинные таблицы
\usepackage{multirow} % Слияние строк в таблице
\usepackage{upgreek}
\usepackage{enumerate}
\usepackage{ dsfont }

%%% Цветной текст

\usepackage[usenames]{color}
\usepackage{colortbl}
\usepackage[table,xcdraw]{xcolor}

%%% Солнышко

\usepackage[weather]{ifsym}

\usepackage{booktabs}
\usepackage{dcolumn}

%%% Гиперссылки

\usepackage{xcolor}
\usepackage{hyperref}
\definecolor{linkcolor}{HTML}{199B03} % цвет ссылок
\definecolor{urlcolor}{HTML}{199B03} % цвет гиперссылок

\hypersetup{pdfstartview=FitH,  linkcolor=linkcolor,urlcolor=urlcolor, colorlinks=true}

\usepackage{minted}

%% Tikz

\usepackage{pgf,tikz,pgfplots}
\pgfplotsset{compat=1.15}
\usepackage{mathrsfs}
\usetikzlibrary{arrows}
\pagestyle{empty}

%%Вставка картинок
\usepackage{graphicx}
\graphicspath{{}}
\DeclareGraphicsExtensions{.pdf,.png,.jpg}


%% эконометрические сокращения
\def \hb{\hat{\beta}}
\DeclareMathOperator{\sVar}{sVar}
\DeclareMathOperator{\sCov}{sCov}
\DeclareMathOperator{\sCorr}{sCorr}


\def \hs{\hat{s}}
\def \hy{\hat{y}}
\def \hY{\hat{Y}}
\def \he{\hat{\varepsilon}}
\def \v1{\vec{1}}
\def \cN{\mathcal{N}}
\def \e{\varepsilon}
\def \z{z}
\def \hb{\hat{\beta}}

\def \hVar{\widehat{\Var}}
\def \hCorr{\widehat{\Corr}}
\def \hCov{\widehat{\Cov}}

\DeclareMathOperator{\tr}{tr}
\DeclareMathOperator*{\plim}{plim}


%% лаг
\renewcommand{\L}{\mathrm{L}}


\usepackage{pgf}
\usepackage{tikz}
\usetikzlibrary{arrows,automata}


% DEFS
\def \mbf{\mathbf}
\def \msf{\mathsf}
\def \mbb{\mathbb}
\def \tbf{\textbf}
\def \tsf{\textsf}
\def \ttt{\texttt}
\def \tbb{\textbb}

\def \wh{\widehat}
\def \wt{\widetilde}
\def \ni{\noindent}
\def \ol{\overline}
\def \cd{\cdot}
\def \bl{\bigl}
\def \br{\bigr}
\def \Bl{\Bigl}
\def \Br{\Bigr}
\def \fr{\frac}
\def \bs{\backslash}
\def \lims{\limits}
\def \arg{{\operatorname{arg}}}
\def \dist{{\operatorname{dist}}}
\def \VC{{\operatorname{VCdim}}}
\def \card{{\operatorname{card}}}
\def \sgn{{\operatorname{sign}\,}}
\def \sign{{\operatorname{sign}\,}}
\def \xfs{(x_1,\ldots,x_{n-1})}
\def \Tr{{\operatorname{\mbf{Tr}}}}
\DeclareMathOperator*{\argmin}{arg\,min}
\DeclareMathOperator*{\argmax}{arg\,max}
\DeclareMathOperator*{\amn}{arg\,min}
\DeclareMathOperator*{\amx}{arg\,max}
\def \cov{{\operatorname{Cov}}}
\DeclareMathOperator{\Var}{Var}
\DeclareMathOperator{\Cov}{Cov}
\DeclareMathOperator{\Corr}{Corr}

\def \xfs{(x_1,\ldots,x_{n-1})}
\def \ti{\tilde}
\def \wti{\widetilde}


\def \mL{\mathcal{L}}
\def \mW{\mathcal{W}}
\def \mH{\mathcal{H}}
\def \mC{\mathcal{C}}
\def \mE{\mathcal{E}}
\def \mN{\mathcal{N}}
\def \mA{\mathcal{A}}
\def \mB{\mathcal{B}}
\def \mU{\mathcal{U}}
\def \mV{\mathcal{V}}
\def \mF{\mathcal{F}}

\def \R{\mbb R}
\def \N{\mbb N}
\def \Z{\mbb Z}
\def \P{\mbb{P}}
%\def \p{\mbb{P}}
\def \E{\mbb{E}}
\def \F{\mbb{F}}
\def \D{\msf{D}}
\def \I{\mbf{I}}
\def \L{\mathcal{L}}

\def \a{\alpha}
\def \b{\beta}
\def \t{\tau}
\def \dt{\delta}
\def \e{\varepsilon}
\def \ga{\gamma}
\def \kp{\varkappa}
\def \la{\lambda}
\def \sg{\sigma}
\def \sgm{\sigma}
\def \tt{\theta}
\def \ve{\varepsilon}
\def \Dt{\Delta}
\def \La{\Lambda}
\def \Sgm{\Sigma}
\def \Sg{\Sigma}
\def \Tt{\Theta}
\def \Om{\Omega}
\def \om{\omega}

\usepackage{graphicx}
\usepackage[referable]{threeparttablex}

%%% Заголовок
\author{Зехов Матвей}
\title{Заметки по многошаговому прогнозированию}
\date{\today}

\begin{document}
	\newpage
\thispagestyle{empty}
\begin{center}
	\textbf{ПРАВИТЕЛЬСТВО РОССИЙСКОЙ ФЕДЕРАЦИИ}\\
	\vspace{2ex}
	\textbf{Федеральное государственное автономное\\ образовательное учреждение высшего образования}
	
	\vspace{2ex}
	
	\textbf{Национальный исследовательский университет \\ <<Высшая школа экономики>>}
	
	\vspace{8ex}
	\begin{flushright}
		Факультет экономических наук\\
		Образовательная программа <<Экономика>>
	\end{flushright}
\end{center}
\vspace{9ex}

\begin{center}
	{\textbf{ДОМАШНЕЕ ЗАДАНИЕ 2
	}}
	\vspace{1ex}
	
	<<Прикладная микроэконометрика>>
\end{center}
\vspace{1ex}
\begin{flushright}
	\noindent
	Студент группы БЭК165\\Зехов Матвей Сергеевич\\
	\vspace{13ex}
	Преподаватель:\\
	Потанин Богдан Станиславович
	
\end{flushright}	

\vfill

\begin{center}
	Москва 2019
	
\end{center}

\newpage
	\tableofcontents
	
\newpage
\section{Теория и гипотезы}
\subsection{}
\Sun 
Выберите независимые переменные. Теоретически обоснуйте выбор каждой из
них. Укажите предполагаемые направления эффектов. При этом вам понадобится как
минимум одна непрерывная переменная и одна дамми переменная (не рекомендуется
брать больше трех различных переменных, не считая их нелинейных преобразований:
квадрат, логарифм, перемножение с целью получения переменной взаимодействия и т.д.).


В качестве независимых переменных возьмём возраст, квадрат возраста, дамми-переменную, демонстрирующую наличие несовершеннолетних детей, а также переменную логарифм переменной uj452.2. Последняя переменная отражает доход, который, по мнению индивида, должна иметь его семья, чтобы он считал, что живёт не хуже других. 

Касательно возраста, очевидно, что предложение труда должно зависеть от возраста индивида, так как с возрастом у него могут изменяться условия. Например, молодёжь, которая ещё получает образование, имеет меньшее предложение труда, а опытные специалисты - большее. Ожидается, что эффект будет положительным. Эффект ниличия несовершеннолетних детей также должен иметь положительный эффект, так как дети требуют, очевидно, больших расходов и индивиду придётся больше работать. Желаемый доход отражает субъективный фактор мотивации. Предполагается, что чем больше денег нужно человеку для того, чтобы быть не хуже других, то тем больше он готов работать. Следственно, эффект будет положительным. Предполагается, что этот показатель входит в уравнение нелинейно, а именно, логарифмически.

\subsection{}

\Sun Сформулируйте по крайней мере одну гипотезу о наличии эффекта
взаимодействия и нелинейного эффекта (например, квадратичного). Теоретически
обоснуйте выдвигаемые вами гипотезы. Включите нелинейные переменные и переменные
взаимодействия в вашу модель.

 Эффект взаимодействия наличия детей и мотивации отражает схожий с предыдущими эффект. Наличие детей будет повышать коэффициент перед мотивацией, так как для тех, кто имеет детей, также понадобится иметь больший доход, чтобы быть не хуже других. Эффект предполагается положительным.  Квадратичный эффект возраста можно объяснить следующим образом. Предположим, что предложение труда зависит параболически от возраста. То есть, что есть точка максимума предложения труда в определённом возрасте, после которой предложение убывает для индивида. Следственно, предполагается, что парабола будет ветвями вниз и эффект будет отрицательным.
 
 \subsection{}
 
\Sun Определитесь с тем, будете ли вы изучать влияние на предложение труда среди
мужчин и женщин в рамках единой модели, либо остановитесь лишь на одном из полов.
Обоснуйте свой выбор теоретически.

Будем рассматривать выборку только совершеннолетних мужчин. Это вызвано тем, что наличие несовершеннолетних детей имеет разнонаправленный эффект на мужчин и женщин, поэтому рассматривать их в рамках одной модели выглядит бессмысленным. Несовершеннолетних индивидов не имеет смысла рассматривать, так как они навряд ли давали репрезентативные ответы о готовности работать. 

\subsection{}

\Sun  Определите границы усечения вашей зависимой переменной и теоретически
обоснуйте их.


Так как в выборке нет ни одного совершеннолетнего мужчины, который бы ответил 0 часов (я сам удивился, но там ответ минимум 6 часов по 5 тысячам наблюдений), то граница усечения будет выше. Будем считать, что индивиды, предлагающие 30 и менее часов - это самозанятые, фрилансеры, безработные, работающие неофициально и тому подобные индивиды. Следовательно, усекать будем по 30 часам.

\section{Обработка данных}

\section{Тобит-модель}

\subsection{}

\Sun Оцените тобит модель, предварительно записав максимизируемую функцию
правдоподобия. Результат представьте в форме таблицы (можно, например, использовать
выдачу из stata, R или python).




\[ L=\prod_{y_{t}=0}\left(1-\Phi\left(\frac{\boldsymbol{x}_{t}^{\prime} \boldsymbol{\beta}}{\sigma}\right)\right) \prod_{y_{z}>0} \frac{1}{\sqrt{2 \pi} \sigma} \exp \left(-\frac{1}{2 \sigma^{2}}\left(y_{t}-\boldsymbol{x}_{t}^{\prime} \boldsymbol{\beta}\right)^{2}\right) \]

\begin{table}[ht]
	\centering
	\begin{tabular}{|rrrrr|}
		\hline
		& Estimate & Std. Error & z value & Pr($>$$|$z$|$) \\ 
		\hline
		(Intercept) & 15.3498 & 9.5342 & 1.6100 & 0.1074 \\ 
		age & 0.4865 & 0.2195 & 2.2165 & 0.0267 \\ 
		I(age\verb|^|2) & -0.0057 & 0.0024 & -2.3929 & 0.0167 \\ 
		child & 23.4039 & 10.4644 & 2.2365 & 0.0253 \\ 
		lmotiv & 1.7636 & 0.6987 & 2.5243 & 0.0116 \\ 
		I(lmotiv * child) & -1.8355 & 0.9058 & -2.0264 & 0.0427 \\ 
		Log(scale) & 2.5198 & 0.0193 & 130.2517 & 0.0000 \\ 
		\hline
	\end{tabular}
\caption{Результаты оценки tobit-модели}
\end{table}


\subsection{}

Как можно видеть из Таблицы 1, все переменные значимы на 5\%-ом уровне, а также за исключением переменной взаиодействия имеют предплолагаемые знаки. Не могу дать содержательнои интерпретации знаку этого коэффициента. 

\subsection{}

\Sun Запишите формулы, по которым можно рассчитать предельные эффекты в
тобит модели для переменной, входящей линейно, в отношении:

А) $ \E(y^*) $

Б) $ \E(y) $

В) Вероятности того, что индивид работает

\begin{enumerate}
	\item \[ \E(Y_i^*) = x_i^\prime \b \]
	\[ \frac{d \E(Y_i^*)}{d x^j} = \b^j \]
	
	\item 	\[ \frac{d \E(Y_i)}{d x^j} = \Phi\left(\dfrac{x_i^\prime \b - 30 }{\sigma}\right)\b^j \]
	
	\item \[ \P\{y^* >  30\} =\Phi\left(\dfrac{x_i^\prime \b - 30 }{\sigma}\right) \]
	\[  \frac{d \P\{y^* >  30\}}{d x^j}  = \phi\left(\dfrac{x_i^\prime \b - 30 }{\sigma}\right)\b^j \]
\end{enumerate}

\subsection{}

\Sun Проинтерпретируйте предельный эффекты для всех независимых переменных
входящих нелинейно или имеющих взаимодействие на $ \E(Y_i^*) $

Так как выписывать формулы никто и не просит, то я и не буду. Расчёты будут для медианного индивида. Его характеристики представлены в Таблице 2, а предельные эффекты - в Таблице 3.

\begin{table}[ht]
	\centering
	\begin{tabular}{|rrr|}
		\hline
		Возраст & Наличие несов. детей & Мотивация \\ 
		\hline
		 44.00 & 1.00 & 11.51 \\ 
		\hline
	\end{tabular}
\caption{{Характеристики медианного индивида}}
\end{table}

\begin{table}[ht]
	\centering
	\begin{tabular}{|rrr|}
		\hline
		Возраст & Наличие несов. детей & Мотивация \\ 
		\hline
		 -0.01 & -18.86 & -1.91 \\ 
		\hline
	\end{tabular}
\caption{Предельные эффекты медианного индивида}
\end{table}

При увеличении возраста медианного индивида на 1 год предложение труда снижается на 0.01 часа в неделю. При наличии несовершеннолетних детей предложение труда снижается на 18.86 часов. При увеличении "мотивации" на 1\% предложение труда снижается на 1.91 часа в неделю. 

\subsection{}
\Sun Проинтерпретируйте предельный эффекты для всех независимых переменных
входящих линейно на
 $ \E(y) $.
 
Линейно входящих переменных нет.


\subsection{}

\Sun Проинтерпретируйте предельный эффекты для всех независимых переменных
на вероятность того, что индивид работает.

Предельные эффекты представлены в Таблице 4. При увеличении возраста на 1 год вероятность занятости уменьшается на 0.00001. При наличии несовершеннолетних детей вероятность уменьшается на 0.016. При увеличении "мотивации" на 1\% вероятность снижается на 0.00167.

\begin{table}[ht]
	\centering
	\begin{tabular}{|rrr|}
		\hline
		 Возраст & Наличие несов. детей & Мотивация \\ 
		\hline
	 -0.00001 & -0.01647 & -0.00167 \\ 
		\hline
	\end{tabular}
\caption{Предельные эффекты вероятности занятости медианного индивида}
\end{table}

\subsection{}
\Sun Предскажите значения
$E\left(y^{*}\right), E(y)$
и вероятности занятости для индивида с
вашими характеристиками, выписав формулу, по которой осуществлялся расчет.

Мои характеристики представлены в Таблице 5. Прогнозные значения представлены в Таблице 6.

Формулы:

$ v $ - вектор моих характеристик
\begin{enumerate}
	\item $ \hat{\E(Y^*)} = v^\prime \hb $
	
	\item $ \hat{\E(Y)} = \Phi\left( \frac{v^\prime \hb - 30}{\hat{\sigma}} \right) v^\prime \hb + \hat{\sigma} \phi\left( \frac{v^\prime \hb - 30}{\hat{\sigma}} \right) $
	
	\item $ \hat{\P\{Y^* > 30\}} = \Phi\left( \frac{v^\prime \hb - 30}{\hat{\sigma}} \right) $
\end{enumerate}
\begin{table}[ht]
	\centering
	\begin{tabular}{|rrrrrrr|}
		\hline
		& Intercept & Возраст & Возраст**2 & Несов. дети & Логарифм з.п. & Дети*Логарифм з.п. \\ 
		\hline
		1 & 1.00 & 21.00 & 441.00 & 0.00 & 12.21 & 0.00 \\ 
		\hline
	\end{tabular}
\caption{Мои характеристики}
\end{table}

\begin{table}[ht]
\centering
\begin{tabular}{|rrr|}
	\hline
	$ \hat{\E(Y^*)} $ & $ \hat{\E(Y)} $ & $ \hat{\P\{Y^* > 30\}} $ \\ 
	\hline
	 44.58 & 39.42 & 0.88 \\ 
	\hline
\end{tabular}
\caption{Прогнозы}
\end{table}

\subsection{}
\Sun Выведите, описав процесс вывода в тексте работы, формулу для расчета
предельного эффекта переменной, входящей нелинейно, на
$ \E(Y) $
Вспомним формулу:

\[  \hat{\E(Y)} = \Phi\left( \frac{x^\prime \hb - 30}{\hat{\sigma}} \right) x^\prime \hb + \hat{\sigma} \phi\left( \frac{x^\prime \hb - 30}{\hat{\sigma}} \right) 
 \]
 
При наших переменных возьмём производную по возрасту, воспользовавшись тем, что $ \phi^\prime = -x\phi $
\[ \frac{d\hat{\E(Y_i)}}{d age} = \left( \frac{\b_1 + 2 \b_2 age_i}{\hat{\sigma}} \right) \phi \left(\frac{x_i^\prime \hb - 30}{\hat{\sigma}} \right) x_i^\prime \hb + \Phi \left(\frac{x_i^\prime \hb - 30}{\hat{\sigma}} \right)   (\b_1 + 2 \b_2 age_i) +\] 

\[  \hat{\sigma} \left(\frac{30 - x_i^\prime \hb}{\hat{\sigma}}\right) \phi \left(\frac{x_i^\prime \hb - 30}{\hat{\sigma}} \right) = \]

\[\phi \left(\gamma_i \right) \left( \left( \frac{\b_1 + 2 \b_2 age_i}{\hat{\sigma}} \right)  x_i^\prime \hb + \hat{\sigma} \left( -\gamma_i \right) + \frac{\b_1 + 2 \b_2 age_i}{\lambda_i} \right) \]

где $ \lambda_i = \frac{\phi \left(\gamma_i\right) }{\Phi \left( \gamma_i \right) }, \gamma_i = \frac{x_i^\prime \hb - 30}{\hat{\sigma}} $




\section{}

\subsection{}
\Sun Проверьте, можно ли исключить из модели эффект взаимодействия,
предварительно записав нулевую гипотезу и статистику теста, а также указав ее
распределение.

\[  H_0: \beta_{lmotiv*child} = 0  \]

\[ H_A: \beta_{lmotiv*child} \neq 0  \]

Оценки ММП асимптотически нормальны, так что гипотеза проверется Z-тестом.  Статистика:

\[ \frac{\hat{\beta}_{lmotiv*child} - 0 }{\hat{\sigma_\beta}} \sim N(0,1)\]

p-value было представлено в Таблице 1. Основная гипотеза отвергается на 5\%-ом уровне.

\subsection{}
\Sun Проверьте, можно ли оценивать совместную модель для тех, кто состоит в браке
и тех, кто нет, предварительно записав нулевую гипотезу и статистику теста, а также указав
ее распределение.

Проверим аналогично LR-тестом.  

$ H_0: Общая модель эквивалентна двум отдельным моделям $

$ H_A: Нельзя перейти к общей модели, нужно оценивать две отдельные. $

Оценим три модели и получим статистику теста:

\[ 2(l_{UR1} + l_{UR2} - l_R) \sim \chi^2_7 \]

Семь степеней свободы, так как каждая модель имеет по 6 коэффициентов и плюс стандартное отклонение. Нулевая гипотеза отвергается на любом разумном уровне значимости.
\subsection{}

\Sun  Опишите, к чему могут приводить, во-первых, гетероскедастичность, а вовторых, нарушение допущения о распределении случайной ошибки в тобит модели


Гетероскедастичность может привести как к неэффективности оценок, так и к несостоятельности, так как если правдободобие будет построено исходя из гомоскедастичности, то будет максимизировано не то правдоподобие. Оценки будут смещёнными и несостоятельными.

При нарушении предпосылки о распределении случайной ошибки опять же, будет максимизировано неверное правдоподобие и, как следствие, смещённые и несостоятельные оценки.

\subsection{}
 Запишите функцию правдоподобия для тобит модели с гетероскедастичной
случайной ошибкой и формально опишите тестирование гипотезы о наличии
гетероскедастичности. Для тех, кто работает в R, дополнительно следует оценить
параметры данной модели используя функцию с семинара и представить результаты в
форме таблицы, а также сделать вывод о наличии или отсутствии гетероскедастичности.

\[ L=\prod_{y_{i}=0}\left(1-\Phi\left(\frac{\boldsymbol{x}_{i}^{\prime} \boldsymbol{\beta}}{\sigma_i}\right)\right) \prod_{y_{i}>30} \frac{1}{\sqrt{2 \pi} \sigma_i} \exp \left(-\frac{1}{2 \sigma_i^{2}}\left(y_{t}-\boldsymbol{x}_{i}^{\prime} \boldsymbol{\beta}\right)^{2}\right) \]
 

Для проверки гипотезы оценим модель с учётом гетероскедастичности и без, и сравним их LR-тестом. Ограничения будут наложены на уравнение, описывающее дисперсию. Количество ограничений будет равно количеству переменных, от которых зависит дисперсия модели. Обозначим эту величину за $ q $.

$ H_0: \sigma_i = \sigma $

$ H_A: \exists \text{ i s.t. } \sigma_i \neq \sigma $

\[ 2(l_{UR} - l_{R}) \sim \chi^2_q \]


Предположим, что гетероскедастичность порождается возрастом. Например, у молодёжи может сильнее колебаться предложение труда в связии с учёбой, личной жизнью, хобби, социальной активностью и прочим. Хотя откуда мне знать про всё, кроме первого, я же в вышке учусь. Гипотеза отвергается на любом разумном уровне значимости. Результаты оценки модели с учётом гетероскедастичности представлены в Таблице 7. Как хорошо видно, коэффициент при возрасте в уравнении дисперсии незначим.

\begin{table}[ht]
	\centering
	\begin{tabular}{|rrrrr|}
		\hline
		& Estimate & Std. Error & z value & Pr($>$$|$z$|$) \\ 
		\hline
		(Intercept) & 15.2810 & 9.4520 & 1.6167 & 0.1059 \\ 
		age & 0.4968 & 0.2195 & 2.2635 & 0.0236 \\ 
		I(age\verb|^|2) & -0.0058 & 0.0024 & -2.4503 & 0.0143 \\ 
		child & 23.0467 & 10.4263 & 2.2104 & 0.0271 \\ 
		lmotiv & 1.7458 & 0.6890 & 2.5336 & 0.0113 \\ 
		I(lmotiv * child) & -1.8017 & 0.9026 & -1.9962 & 0.0459 \\ 
		(scale)\_(Intercept) & 2.5804 & 0.0844 & 30.5908 & 0.0000 \\ 
		(scale)\_age & -0.0014 & 0.0019 & -0.7406 & 0.4589 \\ 
		\hline
	\end{tabular}
\caption{Результаты оценки модели с гетероскедастичностью}
\end{table}


\section{}

\subsection{}

\Sun Оцените две дополнительные модели: с помощью МНК и усеченной регрессии,
представив результаты оценивания в форме таблицы, а также выписав функцию
правдоподобия для усеченной регрессии.


Функция правдоподобия усечённой модели:

\[
\begin{aligned} l=&-\frac{n}{2}\left(\ln (2 \pi)+\ln \sigma^{2}\right)-\frac{1}{2 \sigma^{2}} \sum_{i}\left(y_{i}-\boldsymbol{x}_{i}^{\prime} \boldsymbol{\beta}\right)^{2} -\sum_{i} \ln \left[1-\Phi\left(\frac{30-x_{i}^{\prime} \boldsymbol{\beta}}{\sigma}\right)\right] \end{aligned}
\]

Результаты оценки усечённой модели представлены в Таблице 8. Результаты оценки линейной модели представлены в Таблице 9.

\begin{table}[ht]
	\centering
	\begin{tabular}{|rrrrr|}
		\hline
		& Estimate & Std. Error & z value & Pr($>$$|$z$|$) \\ 
		\hline
		(Intercept) & 2.4252 & 23.5301 & 0.1031 & 0.9179 \\ 
		age & 0.9390 & 0.5778 & 1.6251 & 0.1041 \\ 
		I(age\verb|^|2) & -0.0113 & 0.0063 & -1.7901 & 0.0734 \\ 
		child & 13.2390 & 24.4356 & 0.5418 & 0.5880 \\ 
		lmotiv & 1.3219 & 1.6769 & 0.7883 & 0.4305 \\ 
		I(lmotiv * child) & -0.8876 & 2.1186 & -0.4190 & 0.6752 \\ 
		sigma & 17.6455 & 0.8673 & 20.3444 & 0.0000 \\ 
		\hline
	\end{tabular}
\caption{Результаты оценки усечённой модели}
\end{table}


\begin{table}[ht]
	\centering
	\begin{tabular}{|rrrrr|}
		\hline
		& Estimate & Std. Error & t value & Pr($>$$|$t$|$) \\ 
		\hline
(Intercept) & 15.7208 & 9.6692 & 1.6259 & 0.1042 \\ 
age & 0.4490 & 0.2221 & 2.0222 & 0.0433 \\ 
I(age\verb|^|2) & -0.0053 & 0.0024 & -2.1951 & 0.0283 \\ 
child & 23.4596 & 10.6378 & 2.2053 & 0.0276 \\ 
lmotiv & 1.7817 & 0.7090 & 2.5128 & 0.0121 \\ 
I(lmotiv * child) & -1.8257 & 0.9208 & -1.9827 & 0.0476 \\ 
		\hline
	\end{tabular}
\caption{Результаты оценки линейной модели}
\end{table}



\subsection{}

\Sun Какие оценки являются более эффективными: тобит модели или у усеченной
регрессии. Ответ обоснуйте.

Здесь следует сказать, что в Тобит-модели всё же содержится некоторая информация о усечённых налблюдениях, хотя и искажённая. В усечённой модели же эти данные отсутствуют вовсе, и корректировка происходит исходя из предположений о распредении данных. Следовательно, первые оценки будут более эффективными, так как они учитывают болльше информации. Это видно из результатов оценки. В усечённой регресси большая часть переменных стала незначимой.

\subsection{}
\Sun Выберите лучшую модель на основании критерия AIC.

По AIC нет смысла сравнивать, так как модели оценены на разном количестве наблюдений.

\subsection{}

\Sun Выберите лучшую модель по критерию MSE, посчитанному на тестовой
выборке

Отведём на тестовую часть 0.2 выборки и переоценим модели на оставшихся данных. По результатам вневыборочного прогноза из Таблицы 10 можно сказать, что усечённая модель даёт слишком смещённые прогнозы. Линейная и Tobit-модели почти эквивалентны, но линейная слегка лучше. Её и признаем оптимальной.

\begin{table}[ht]
	\centering
	\begin{tabular}{|rrrr|}
		\hline
		& Усечённая & Линейная & Tobit \\ 
		\hline
		 & 241.1 & 153.9 & 154.2 \\ 
		\hline
	\end{tabular}
\caption{MSE вневыборочного прогноза}
\end{table}
\end{document}